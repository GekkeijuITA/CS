\documentclass[12pt, a4paper]{report}
\usepackage[utf8]{inputenc}
\newcommand\preamble{
    \usepackage[italian]{babel}
    \usepackage{geometry}
    \usepackage{amsmath}
    \usepackage{amssymb}
    \usepackage{graphicx}
    \usepackage{ulem}
    \geometry{margin=2cm}
    \usepackage{listings}
    \usepackage{xcolor}
    \let\olditemize\itemize
    \renewcommand\itemize{\olditemize\setlength\itemsep{0em}}
}
% Definizione delle variabili
\newcommand{\imagePath}{Immagini/logoUni.png}

% Definizione del comando per la pagina di titolo con argomenti
\newcommand{\customTitlePage}[2]{
    \newcommand{\courseTitle}{#1}
    \newcommand{\academicYear}{#2}
    
    \begin{titlepage}
        \centering
        \includegraphics[width=0.5\textwidth]{\imagePath}\par\vspace{1cm}
        {\scshape\LARGE University of Studies of Genoa \par}
        \vspace{1.5cm}
        {\huge\bfseries \courseTitle \par}
        \vspace{2cm}
        {\Large\itshape Lorenzo Vaccarecci \par}
        \vfill
        \academicYear
    \end{titlepage}
}

\preamble

\begin{document}
\customTitlePage{Computer Security}{2024-2025}
\newpage
\tableofcontents
\chapter{Introduzione}
\section{Information Security}
\begin{itemize}
    \item La sicurezza concerne la protezione degli \textbf{asset} dalle \textbf{minacce (threats)}
    \item I proprietari (\textbf{owners}) valorizzano i loro asset e vogliono  proteggerli
    \item I proprietari analizzano le minacce e valutano i rischi. Questo aiuta la selezione di \textbf{contromisure} che riducono le \textbf{vulnerabilità}
\end{itemize}
\begin{equation*}
    Risk_{E} = P(E) \cdot I_{E}
\end{equation*}
Dove $E$ è l'evento che rappresenta la minaccia, $P(E)$ è la probabilità che l'evento si verifichi e $I_{E}$ è l'impatto che l'evento ha.
\begin{equation*}
    Risk_{Tot} = \sum_{e\in E} (P(e) \cdot I_{e})
\end{equation*}
$P(\cdot)$ può essere:
\begin{itemize}
    \item 0.7 - 1 : Alta
    \item 0.4 - 0.7 : Media
    \item $\leq 0.3$ : Bassa
\end{itemize}
\section{Security Properties}
\begin{itemize}
    \item Confidentiality: l'informazione non è conosciuta da non autorizzati, bisogna permettere solo a chi ne ha diritto attraverso \textbf{security policies}. Qualche volta si dice \textbf{privacy} per gli individui, \textbf{secrecy} per le organizzazioni, \textbf{anonymity} invece per nascondere l'identità.
    \item Integrity: l'informazione non deve essere modificata in modo malizioso.
    \item Authentication: i dati o i servizi devono essere accessibili solo da chi autorizzato. Solitamente il metodo di autenticazione è qualcosa che si ha, qualcosa che si conosce o qualcosa che sei (impronta digitale, firma, biometrica).
    \item Availability: i dati o i servizi devono esere accessibili e utilizzabili in qualsiasi momento. Questo significa che bisogna prevenire da attacchi DoS (\textbf{Denial of Service})
    \item Accountability: le azioni sono registrare e rintracciabili dalle parti responsabili.
\end{itemize}
\section{Protection Countermeasures}
\begin{itemize}
    \item Prevention: prevenire gli attacchi attraverso la progettazione di sistemi e impiegando tecnologie di sicurezza.
    \item Detection: i metodi principali sono il \textbf{logging} e il \textbf{MACs} (file hash per rilevare alterazioni).
    \item Response: varia dal ripristinare backup all'informare le autorità competenti o le parti coinvolte.
    \item Remediation
\end{itemize}
\section{Managing security: implementing a solution}
\begin{itemize}
    \item Security Analysis: analizza le minacce che potrebbero compromettere l'asset e propone delle politiche e soluzioni a costi appropriati.
    \item Threat Model: documenta le possibile minacce al sistema, immaginando tutte le possibili vulnerabilità che possono essere sfruttate.
    \item Risk Assessment: valutazione quantitativa dei rischi.
    \item Security Policy: per ogni rischio si descrivono le contromisure da adottare.
    \item Security Solution: progettazione e implementazione delle tecnologie appropriate a costi appropriati.
\end{itemize}
\end{document}